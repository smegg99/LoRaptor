\subsection{Cele, założenia i zastosowania}

LoRaptor to projekt powstały z potrzeby zapewnienia niezależnej komunikacji bezprzewodowej w sytuacjach, gdzie tradycyjne sieci zawodzą --- w terenie, w warunkach kryzysowych, podczas biwaków, wypraw czy gier terenowych.
Naszym celem było stworzenie lekkiego, energooszczędnego urządzenia, które:
\begin{itemize}
	\item umożliwia komunikację bez potrzeby dostępu do Wi-Fi, GSM czy Internetu,
	\item tworzy dynamiczną sieć mesh opartą na technologii LoRa,
	\item działa intuicyjnie i bez konfiguracji sieciowej,
	\item współpracuje z aplikacją mobilną oraz terminalem CLI,
	\item może pełnić rolę płytki deweloperskiej.
\end{itemize}
Główne zastosowania LoRaptora:
\begin{itemize}
	\item komunikacja awaryjna i terenowa,
	\item działania survivalowe i gry outdoorowe,
	\item projekty edukacyjne, hobbystyczne i IoT,
	\item eksperymenty z sieciami bezprzewodowymi i protokołem LoRa.
\end{itemize}
Dzięki prostej budowie, łatwej integracji oraz dużej elastyczności, LoRaptor stanowi solidne narzędzie zarówno dla użytkowników końcowych, jak i dla twórców rozwiązań technologicznych.