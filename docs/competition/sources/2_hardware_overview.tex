\subsection{Przebieg pracy}

Pomysł stworzenia systemu niezależnej komunikacji LoRa pojawił się na początku 2024 roku jako odpowiedź na rosnące zapotrzebowanie na niezależne, bezpieczne i funkcjonalne rozwiązania komunikacyjne. Inspiracją były zarówno projekty open-source, jak i własne doświadczenia z pracy w terenie oraz zamiłowanie do mikrokontrolerów ESP32.

Początkowo projekt istniał wyłącznie jako koncepcja. W połowie 2024 roku rozpoczęły się pierwsze konkretne działania:

\subsection*{Faza I --- Eksperymenty i testy podstawowe}

\begin{itemize}
	\item Testy zasięgu modułów LoRa (RA-02) i wybór parametrów urządzenia.
	\item Pierwsze połączenia ESP32 z LoRa za pomocą interfejsu SPI.
	\item Próby z bibliotekami Arduino i eksperymenty z transmisją tekstu.
\end{itemize}

\subsection*{Faza II --- Budowa sprzętu}

\begin{itemize}
	\item Projekt i montaż płytki PCB z modułem ESP32-S3-MINI-1.
	\item Zintegrowanie diody RGB i złącz USB-C.
	\item Testy zasilania i eliminacja zakłóceń w komunikacji.
\end{itemize}

\clearpage
\subsection*{Faza III --- Rozwój oprogramowania}

\begin{itemize}
	\item Stworzenie własnej biblioteki do obsługi wiersza poleceń --- \texttt{RaptorCLI}.
	\item Wykorzystanie biblioteki \texttt{LoRaMesher} jako podstawy do komunikacji mesh.
	\item Opracowanie kompresji wiadomości za pomocą algorytmu SMAZ i szyfrowania AES (sprzętowe).
	\item Integracja komunikacji BLE oraz USB CDC.
\end{itemize}

\subsection*{Faza IV --- Aplikacja mobilna}

\begin{itemize}
	\item Zaprojektowanie aplikacji \texttt{RaptChat} w języku Dart (Flutter).
	\item Obsługa połączeń przez BLE i USB z widokiem czatu.
	\item Testy interfejsu użytkownika i poprawa responsywności.
\end{itemize}

\subsection*{Faza V --- Integracja i testy końcowe}

\begin{itemize}
	\item Połączenie wszystkich modułów sprzętowych i programowych w spójną całość.
	\item Testy funkcjonalne w warunkach zbliżonych do rzeczywistych.
	\item Projekt i wydruk obudowy 3D z litofanem (oknem sygnalizacyjnym).
\end{itemize}
\smallskip
Od pomysłu do wykonania minął ponad rok --- od pierwszych szkiców i prób z przewodami na płytce stykowej, aż po działające urządzenie w obudowie, gotowe do pracy w terenie. Dziś, w marcu 2025 roku, LoRaptor stanowi w pełni działający prototyp, który spełnia założone cele i może być dalej rozwijany.