\subsection{Aplikacja RaptChat}

RaptChat to mobilna aplikacja napisana w języku \textbf{Dart} z użyciem frameworka \textbf{Flutter}, przeznaczona do zarządzania urządzeniem LoRaptor i prowadzenia bezprzewodowej komunikacji tekstowej za pośrednictwem sieci mesh. Aplikacja umożliwia użytkownikowi:

\begin{itemize}
	\item wyszukiwanie i łączenie się z urządzeniami LoRaptor przez BLE (Bluetooth Low Energy),
	\item konfigurację połączeń,
	\item wysyłanie oraz odbieranie wiadomości,
	\item wizualizację aktualnych węzłów w sieci,
	\item zarządzanie lokalnymi ustawieniami i urządzeniami.
\end{itemize}

\vspace{1cm}
\begin{figure}[H]
    \centering
    \begin{minipage}[b]{0.45\textwidth}
        \centering
        \includegraphics[width=\textwidth]{root/app_logo_1.png}
        \caption{Główne logo RaptChat}
    \end{minipage}
    \hfill
    \begin{minipage}[b]{0.45\textwidth}
        \centering
        \includegraphics[width=\textwidth]{root/app_logo_2.png}
        \caption{Alt. logo RaptChat}
    \end{minipage}
\end{figure}

\clearpage
\subsubsection{Architektura działania}

Aplikacja łączy się z urządzeniem za pomocą profilu \textbf{NUS (Nordic UART Service)}, który zapewnia kanał tekstowej komunikacji BLE. Dane przesyłane przez aplikację są komendami w formacie zdefiniowanym przez bibliotekę \texttt{RaptorCLI}. Aplikacja posiada własny parser CLI (dispatcher), który umożliwia zrozumienie odpowiedzi urządzenia i podejmowanie akcji na podstawie ich typu.

\clearpage
\subsubsection{Komunikacja z urządzeniem}

\begin{itemize}
	\item Komendy są wysyłane tekstowo, np.:
		\begin{lstlisting}
send -id "test123" -m "Czesc!"
		\end{lstlisting}
	\item Odpowiedzi są przetwarzane w czasie rzeczywistym i klasyfikowane według ich typu, do przykładowych typów należą:
		\begin{itemize}
			\item \texttt{msg.send.success} --- potwierdzenie wysłania wiadomości,
			\item \texttt{msg.conn.created} --- potwierdzenie utworzenia połączenia,
			\item \texttt{type.flush.mess} --- wiadomości odebrane z sieci,
			\item \texttt{type.list.nodes} --- lista aktualnie znanych węzłów mesh.
		\end{itemize}
\end{itemize}

\subsubsection{Warstwa BLE}

Za zarządzanie połączeniem Bluetooth odpowiada klasa \texttt{BleDeviceManager}. Obsługuje ona:
\begin{itemize}
	\item skanowanie i filtrowanie urządzeń LoRaptor,
	\item parowanie, łączenie i rozłączanie urządzeń,
	\item inicjalizację kanału komunikacji NUS,
	\item konfigurację początkową urządzenia (RTC, połączenia itp.).,
	\item przesyłanie i odbieranie komend.
\end{itemize}

\clearpage
\subsubsection{Automatyzacja konfiguracji}

Po każdym połączeniu z urządzeniem aplikacja wykonuje tzw. komendy startowe:
\begin{enumerate}
	\item Ustawienie czasu RTC na podstawie zegara telefonu:
		\begin{lstlisting}
set rtc -t 1711000000
		\end{lstlisting}
	\item Odtworzenie wcześniej zapisanych połączeń i ich odbiorców:
		\begin{lstlisting}
create connection -id "test" -k "key1234" -r [53052, 16888]
		\end{lstlisting}
\end{enumerate}

\subsubsection{Warstwa wiadomości}

Logiką wiadomości zarządza \texttt{MessagesManager}, który:
\begin{itemize}
	\item zarządza wysyłaniem wiadomości i kolejką poleceń,
	\item cyklicznie odpytuje urządzenie o nowe wiadomości (komenda \texttt{flush}),
	\item aktualizuje interfejs użytkownika po nadejściu nowych danych.
\end{itemize}

\subsubsection{Interfejs użytkownika}

Aplikacja składa się z kilku ekranów:
\begin{itemize}
	\item \textbf{Ekran główny} — lista zapisanych połączeń i dostęp do funkcji czatu.
	\item \textbf{Czat} — dwukierunkowa komunikacja tekstowa z danym połączeniem.
	\item \textbf{Mapa sieci} — wizualizacja połączonych węzłów mesh.
	\item \textbf{Edycja połączenia} — konfiguracja ID, klucza prywatnego i listy odbiorców.
	\item \textbf{Ustawienia} — język aplikacji, motyw kolorystyczny itp.
\end{itemize}

\begin{figure}[H]
	\centering
	\begin{minipage}[b]{0.45\textwidth}
		\centering
		\includegraphics[width=\textwidth]{root/raptchat_home_dark.png}
		\caption{Ekran główny (tryb ciemny)}
	\end{minipage}
	\hfill
	\begin{minipage}[b]{0.45\textwidth}
		\centering
		\includegraphics[width=\textwidth]{root/raptchat_home.png}
		\caption{Ekran główny (tryb jasny)}
	\end{minipage}
\end{figure}

\clearpage
\begin{figure}[H]
	\centering
	\begin{minipage}[b]{0.45\textwidth}
		\centering
		\includegraphics[width=\textwidth]{root/raptchat_settings.png}
		\caption{Panel ustawień}
	\end{minipage}
	\hfill
	\begin{minipage}[b]{0.45\textwidth}
		\centering
		\includegraphics[width=\textwidth]{root/raptchat_devices_alt.png}
		\caption{Widok urządzeń}
	\end{minipage}
\end{figure}

\clearpage
\begin{figure}[H]
	\centering
	\begin{minipage}[b]{0.45\textwidth}
		\centering
		\includegraphics[width=\textwidth]{root/raptchat_devices.png}
		\caption{Widok urządzeń ze sparowanym LoRaptorem}
	\end{minipage}
	\hfill
	\begin{minipage}[b]{0.45\textwidth}
		\centering
		\includegraphics[width=\textwidth]{root/raptchat_my_device.png}
		\caption{Szczegóły mojego urządzenia}
	\end{minipage}
\end{figure}

\clearpage
\begin{figure}[H]
	\centering
	\begin{minipage}[b]{0.45\textwidth}
		\centering
		\includegraphics[width=\textwidth]{root/raptchat_edit_connection.png}
		\caption{Tworzenie nowego połączenia}
	\end{minipage}
	\hfill
	\begin{minipage}[b]{0.45\textwidth}
		\centering
		\includegraphics[width=\textwidth]{root/raptchat_edit_connection_completed.png}
		\caption{Uzupełnione dane połączenia}
	\end{minipage}
\end{figure}

\clearpage
\begin{figure}[H]
	\centering
	\begin{minipage}[b]{0.45\textwidth}
		\centering
		\includegraphics[width=\textwidth]{root/raptchat_chats_home.png}
		\caption{Lista aktywnych czatów}
	\end{minipage}
	\hfill
	\begin{minipage}[b]{0.45\textwidth}
		\centering
		\includegraphics[width=\textwidth]{root/raptchat_empty_chat.png}
		\caption{Pusty czat (nowa konwersacja)}
	\end{minipage}
\end{figure}

\clearpage
\begin{figure}[H]
	\centering
	\begin{minipage}[b]{0.45\textwidth}
		\centering
		\includegraphics[width=\textwidth]{root/raptchat_chat_2.png}
		\caption{Konwersacja tekstowa (przykład 1)}
	\end{minipage}
	\hfill
	\begin{minipage}[b]{0.45\textwidth}
		\centering
		\includegraphics[width=\textwidth]{root/raptchat_chat.png}
		\caption{Konwersacja tekstowa (przykład 2)}
	\end{minipage}
\end{figure}

\begin{figure}[H]
	\centering
	\begin{minipage}[b]{0.45\textwidth}
		\centering
		\includegraphics[width=\textwidth]{root/raptchat_qr.png}
		\caption{Kod QR do szybkiego dodania połączenia}
	\end{minipage}
\end{figure}

\clearpage
\subsubsection{Podsumowanie}

RaptChat stanowi wygodny interfejs do zarządzania LoRaptorem, bez potrzeby używania terminala lub pisania skryptów. Dzięki rozbudowanej warstwie CLI wbudowanej w urządzenie, aplikacja zyskuje elastyczność, a użytkownik ma pełną kontrolę nad siecią i komunikacją.