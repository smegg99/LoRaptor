\subsection{Off the grid, on the mesh – co to znaczy?}

Hasło \textbf{„Off the grid"} odnosi się do działania poza tradycyjną infrastrukturą sieciową --- bez dostępu do Wi-Fi, sieci komórkowej, internetu czy zasilania z sieci energetycznej. To stan całkowitej niezależności od zewnętrznych systemów.

Z kolei \textbf{„On the mesh"} wskazuje na działanie w ramach sieci mesh, czyli samodzielnie organizującej się sieci urządzeń, które komunikują się między sobą bez centralnego punktu. Topologia siatki (mesh) oznacza, że każde urządzenie może przekazywać dane do wielu innych urządzeń, tworząc elastyczną strukturę o wielu możliwych ścieżkach komunikacji, co zwiększa odporność sieci na awarie poszczególnych węzłów oraz zwiększa jej zasięg.

\begin{tcolorbox}[
	colback=gray!5!white, 
	colframe=gray!75!black, 
	boxrule=0.8pt, 
	arc=5pt,
	enhanced,
	drop shadow,
	top=8pt,
	bottom=8pt,
	center
]
	\begin{center}
		\emph{„Bez internetu, bez zasięgu, bez problemu --- bo nasza sieć działa sama."}
	\end{center}
\end{tcolorbox}
To idealne podsumowanie idei \textbf{LoRaptora}:
\begin{itemize}
	\item Urządzenia działają tam, gdzie nie ma infrastruktury;
	\item Tworzą własną, inteligentną i odporną sieć mesh;
	\item Pozwalają na komunikację prawie wszędzie --- niezależnie od warunków;
\end{itemize}

\vspace{1cm}
\begin{center}
\includegraphics[height=2.5cm]{root/LoRaptorHeader.png}
\end{center}