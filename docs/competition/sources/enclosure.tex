\subsection{Obudowa urządzenia}

Aby LoRaptor mógł funkcjonować jako w pełni przenośne urządzenie terenowe, konieczne było zaprojektowanie dedykowanej obudowy, chroniącej elektronikę przed uszkodzeniami mechanicznymi, a zarazem umożliwiającej łatwą obsługę i estetyczną prezentację projektu.

Obudowa została wykonana we własnym zakresie — zaprojektowana od podstaw w środowisku \textbf{Autodesk Inventor} (licencja studencka), a następnie wydrukowana na zmodyfikowanej drukarce \textbf{Creality Ender 3}, przystosowanej do pracy z filamentami technicznymi.

\vspace{1cm}
\begin{figure}[htbp]
\centering
	\includegraphics[width=0.85\textwidth]{root/in_printing.jpg}
	\caption{Góra obudowy w trakcie drukowania na drukarce 3D}
\end{figure}

\clearpage
Do druku wykorzystano filament \textbf{PETG} polskiej firmy \textbf{FilHub}, w dwóch kolorach:
\begin{itemize}
	\item \textbf{Fioletowy} — jako kolor bazowy, stanowiący główny korpus obudowy,
	\item \textbf{Mleczny biały} — zastosowany w górnej części obudowy jako element dekoracyjno-funkcjonalny.
\end{itemize}

Materiał PETG został wybrany nieprzypadkowo — charakteryzuje się on doskonałym połączeniem wytrzymałości mechanicznej oraz odporności termicznej, znacznie przewyższając w tym zakresie standardowy filament PLA. Dodatkowo, PETG cechuje się zwiększoną odpornością na promieniowanie UV, co ma kluczowe znaczenie dla urządzenia przeznaczonego do pracy w terenie.

Elementy obudowy zostały wydrukowane na szklanym stole, co zapewniło pierwszej warstwie idealnie gładką powierzchnię. Jest to szczególnie istotne w przypadku górnej pokrywy, gdzie zastosowano efekt \textbf{litofanu} (ang. \textit{litophane}) — trójwymiarowej grafiki wygenerowanej w sposób umożliwiający podświetlenie jej od tyłu. Wykorzystując przezroczystość mlecznego PETG, w górnej pokrywie obudowy umieszczono logo projektu, które podświetlane jest od wewnątrz przez diodę LED RGB. Dioda ta pełni funkcję wskaźnika stanu urządzenia, przez co efekt litofanu jest nie tylko dekoracyjny, ale i funkcjonalny.

Całkowity czas wydruku kompletnej obudowy wyniósł około 3-4 godzin. Warto podkreślić, że dzięki zastosowaniu szklanego stołu uzyskano niemal lustrzaną powierzchnię materiału, co dodatkowo podnosi walory estetyczne całej konstrukcji.

\clearpage
Całość została zaprojektowana tak, aby możliwy był bezproblemowy dostęp do portów USB-C oraz zapewniona cyrkulacja powietrza w razie dłuższej pracy urządzenia. W strukturze obudowy zastosowano charakterystyczny wzór heksagonalny (plaster miodu), który nie tylko podniósł walory estetyczne, ale przede wszystkim zwiększył sztywność konstrukcji przy jednoczesnym ograniczeniu zużycia materiału. Taka struktura komórkowa pozwala na optymalne rozłożenie naprężeń mechanicznych, zwiększając odporność obudowy na odkształcenia, przy zachowaniu lekkości całej konstrukcji oraz zmniejszonej ilości wykorzystanego filamentu.

\vspace{1cm}

\begin{figure}[htbp]
\centering
	\includegraphics[width=0.85\textwidth]{root/cover_back.png}
	\caption{Górna część obudowy z funkcjonalnymi otworami wentylacyjnymi w strukturze heksagonalnej}
\end{figure}

\begin{figure}[htbp]
\centering
	\includegraphics[width=0.85\textwidth]{root/cover_top.png}
	\caption{Widok z góry na górną część obudowy}
	\vspace{1cm}
	\includegraphics[width=0.85\textwidth]{root/cover_top_alt.png}
	\caption{Perspektywiczne ujęcie górnej części obudowy ukazujące proporcje i kształt elementów}
\end{figure}

\clearpage
\begin{figure}[htbp]
\centering
	\includegraphics[width=0.85\textwidth]{root/cover_top_closeup_back.png}
	\caption{Zbliżenie na tylną część obudowy prezentujące precyzję łączeń i wykonania elementów konstrukcyjnych}
	\vspace{1cm}
	\includegraphics[width=0.85\textwidth]{root/cover_top_front.png}
	\caption{Front górnej części obudowy}
\end{figure}

\clearpage
\begin{figure}[htbp]
\centering
	\includegraphics[width=0.85\textwidth]{root/cover_top_wireframe.png}
	\caption{Wizualizacja szkieletowa górnej części obudowy}
	\vspace{1cm}
	\includegraphics[width=0.85\textwidth]{root/cover_top_wireframe_front.png}
	\caption{Frontowy widok szkieletowy górnej cześci obudowy}
\end{figure}

\clearpage
\begin{figure}[htbp]
\centering
	\includegraphics[width=0.85\textwidth]{root/enclosure_back.png}
	\caption{Widok tylnej strony dolnej obudowy z charakterystycznym wzorem wentylacyjnym}
	\vspace{1cm}
	\includegraphics[width=0.85\textwidth]{root/enclosure_front.png}
	\caption{Widok frontalny dolnej części obudowy}
\end{figure}

\clearpage
\begin{figure}[htbp]
\centering
	\includegraphics[width=0.85\textwidth]{root/enclosure_closeup_wireframe.png}
	\caption{Zbliżenie na szkieletową strukturę dolnej części obudowy ukazujące detale mocowań}
	\vspace{1cm}
	\includegraphics[width=0.85\textwidth]{root/enclosure_top_closeup.png}
	\caption{Zbliżenie na dolną część obudowy z widocznymi elementami mocujacymi płytkę drukowaną}
\end{figure}

\clearpage
\begin{figure}[htbp]
\centering
	\includegraphics[width=0.85\textwidth]{root/enclosure_side.png}
	\caption{Profil boczny dolnej części obudowy}
\end{figure}

\begin{tcolorbox}[
	colback=gray!5!white, 
	colframe=gray!75!black, 
	boxrule=0.8pt, 
	arc=5pt,
	enhanced,
	drop shadow,
	top=8pt,
	bottom=8pt,
	center
]
\textit{Obudowa stanowi ważne uzupełnienie projektu — łączy funkcję ochronną z estetyką i personalizacją, a efekt litofanu nadaje urządzeniu wyjątkowy charakter.}
\end{tcolorbox}

\clearpage
\subsection{Szacunkowy koszt jednej jednostki}

Na podstawie faktur oraz kosztów zamówienia PCB oszacowano orientacyjny koszt produkcji jednej sztuki LoRaptora:
\begin{itemize}
	\item \textbf{Płytka PCB (JLCPCB)} — ok. 0,24 USD;
	\item \textbf{Wysyłka i cło} — ok. 0,45 USD;
	\item \textbf{Komponenty z TME} — ok. 6,50–7,00 USD;
	\item \textbf{RA-02 + antena} — ok. 3,50 USD;
	\item \textbf{Złącze USB-C (żeńskie)} — ok. 0,30–0,50 USD (Aliexpress);
\end{itemize}
\textbf{Łączny koszt produkcji jednej sztuki: ok. 10–11 USD}  
(około 40–45 złotych, w zależności od kursu walut i dostawców).

\begin{tcolorbox}[
	colback=gray!5!white, 
	colframe=gray!75!black, 
	boxrule=0.8pt, 
	arc=5pt,
	enhanced,
	drop shadow,
	top=8pt,
	bottom=8pt,
	center
]
\textit{Zastosowanie komponentów z TME oraz tanich zamienników z Aliexpress pozwoliło utrzymać niski koszt jednostkowy, przy zachowaniu funkcjonalności niezbędnej w testach i prototypowaniu.}
\end{tcolorbox}