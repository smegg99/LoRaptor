\section{Część sprzętowa}

Projekt \textbf{LoRaptor} wymagał stworzenia dedykowanej, kompaktowej i energooszczędnej platformy sprzętowej, która nie tylko umożliwi bezprzewodową komunikację dalekiego zasięgu, ale będzie również łatwa do integracji z urządzeniami końcowymi — smartfonami, komputerami czy innymi mikrokontrolerami. 

Kluczowym elementem konstrukcji stała się zaprojektowana od podstaw płytka drukowana (PCB), zawierająca dwa główne moduły:
\begin{itemize}
	\item \textbf{ESP32-S3-MINI-1-N8} --- nowoczesny mikrokontroler firmy Espressif z obsługą BLE, Wi-Fi oraz USB-OTG, wyposażony w dużą moc obliczeniową i niskie zużycie energii.
	\item \textbf{RA-02 LoRa} (AIThinker) --- wydajny moduł komunikacyjny oparty na technologii LoRa, pracujący na częstotliwości 433 MHz.
\end{itemize}
Oba moduły zostały połączone na jednej płytce, zaprojektowanej w środowisku \textbf{KiCad}, z myślą o maksymalnym wykorzystaniu przestrzeni i uproszczeniu produkcji.

\clearpage
\subsection{Projekt PCB i montaż komponentów}

Projekt płytki przewiduje:
\begin{itemize}
	\item Port USB-C (męski i żeński) z obsługą komunikacji i zasilania;
	\item Komunikację LoRa poprzez moduł RA-02;
	\item Sygnalizację stanu pracy za pomocą diody RGB;
	\item Przyciski BOOT i RESET dla konfiguracji systemu;
	\item Zabezpieczenia przed ESD i przepięciami;
	\item Filtrację zasilania z wykorzystaniem koralików ferrytowych i kondensatorów;
	\item Do stabilizacji napięcia zastosowano układ LDI1117-3.3H, zapewniający niezawodne zasilanie 3.3V dla wszystkich komponentów.
\end{itemize}

Na płytce umieszczono również złącza szpilkowe (raster 1.27 mm) przeznaczone do programowania, debugowania oraz rozszerzeń funkcjonalnych (wyprowadzono niewykorzystane piny GPIO). Płytka ma grubość 0.8 mm, co pozwala na zastosowanie jej w obudowach o niewielkiej głębokości.

\clearpage
\subsection{Schemat elektroniczny}

Poniżej znajduje się schemat elektryczny w wersji V1.0.0, przedstawiający pełne połączenia między komponentami, pinout ESP32, zasilanie i logikę sterowania.

\begin{figure}[htbp]
\centering
	\includegraphics[width=\textwidth]{root/pcb_schematic.png}
	\caption{Schemat elektroniczny płytki}
\end{figure}

\clearpage
\begin{figure}[htbp]
\centering
	\includegraphics[width=0.8\textwidth]{root/pcb_layout.png}
	\caption{Układ płytki}
	\vspace{1cm}
	\includegraphics[width=0.8\textwidth]{root/pcb_layout_bottom.png}
	\caption{Układ płytki od spodu}
\end{figure}

\clearpage

\begin{figure}[htbp]
\centering
	\includegraphics[width=0.9\textwidth]{root/pcb_render_top.png}
	\caption{Widok 3D góry płytki}
	\vspace{1cm}
	\includegraphics[width=0.9\textwidth]{root/pcb_render_bottom.png}
	\caption{Widok 3D spodu płytki}
\end{figure}
	
\clearpage
\subsection{Produkcja PCB}

Płytka została wykonana w lutym 2025 roku w chińskiej firmie \textbf{JLCPCB} jako seria prototypowa. Zamówienie obejmowało:
\begin{itemize}
	\item 75 sztuk płytek (mała seria),
	\item 1 szablon SMT do nakładania pasty lutowniczej,
	\item całkowity koszt z wysyłką i opłatami: \textbf{\$59.77}.
\end{itemize}
Czas realizacji produkcji wynosił około 5-6 dni + dostawa UPS Express Saver.
\begin{figure}[htbp]
\centering
	\includegraphics[width=0.9\textwidth]{root/pcb_order_screenshot.png}
	\caption{Zamówienie produkcyjne w JLCPCB (75 płytek + stencil SMT)}
\end{figure}

\clearpage
\begin{figure}[htbp]
	\centering
	\includegraphics[width=0.8\textwidth]{root/pcb_batch.jpg}
	\caption{Płytki drukowane po otrzymaniu z produkcji}
\end{figure}

\clearpage
\begin{figure}[htbp]
	\centering
	\includegraphics[width=0.8\textwidth]{root/pcb_stencil.jpg}
	\caption{Stencil SMT do nakładania pasty lutowniczej}
\end{figure}

\clearpage
\subsection{Lista elementów (BOM)}

Pełna lista elementów zastosowanych na płytce znajduje się poniżej. Wszystkie komponenty dobrano tak, aby zapewnić kompatybilność z montażem SMT oraz łatwość ręcznego lutowania.

Zastosowano popularne obudowy SMD (0805 / 1206) i sprawdzone układy o dobrej dostępności.

\begin{itemize}
	\item \textbf{ESP32-S3-MINI-1-N8} — główny mikrokontroler z BLE, Wi-Fi i USB;
	\item \textbf{RA-02 LoRa} — moduł komunikacji 433 MHz z anteną (Aliexpress, ok. 15 zł / szt.);
	\item \textbf{LTST-G563ZEGBW} — dioda RGB LED;
	\item \textbf{1825910-6} — przyciski BOOT i RESET;
	\item \textbf{LDI1117-3.3H} — stabilizator napięcia 3.3V;
	\item \textbf{SMF5.0A}, \textbf{PESD5V0S1UB.115}, \textbf{BAT60A} — zabezpieczenia ESD i przepięciowe;
	\item \textbf{MF-MSMF110} — bezpiecznik polimerowy;
	\item \textbf{USB-C (żeńskie)} — generyczne złącza zakupione na Aliexpress (nierekomendowane do finalnej produkcji, ale wystarczające do prototypów);
	\item Rezystory i kondensatory (5.1kΩ, 10kΩ, 68Ω, 10Ω, 100nF, 10uF, 47uF) – standardowe elementy pasywne SMD;
\end{itemize}