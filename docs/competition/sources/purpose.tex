\subsection{Cele, założenia i zastosowania}

\textbf{LoRaptor} to projekt powstały z potrzeby zapewnienia niezależnej komunikacji bezprzewodowej w sytuacjach, gdzie tradycyjne sieci zawodzą --- w terenie, w warunkach kryzysowych, podczas biwaków, wypraw czy gier terenowych.
Naszym celem było stworzenie lekkiego, energooszczędnego urządzenia, które:
\begin{itemize}
	\item Umożliwia komunikację bez potrzeby dostępu do Wi-Fi, GSM czy Internetu;
	\item Tworzy dynamiczną sieć mesh opartą na technologii LoRa;
	\item Działa intuicyjnie i bez konfiguracji sieciowej;
	\item Współpracuje z aplikacją mobilną oraz terminalem CLI;
	\item Może pełnić rolę płytki deweloperskiej;
\end{itemize}
Główne zastosowania \textbf{LoRaptora}:
\begin{itemize}
	\item Komunikacja awaryjna i terenowa;
	\item Działania survivalowe i gry outdoorowe;
	\item Projekty edukacyjne, hobbystyczne i IoT;
	\item Eksperymenty z sieciami bezprzewodowymi i protokołem LoRa;
\end{itemize}
Dzięki prostej budowie, łatwej integracji oraz dużej elastyczności, \textbf{LoRaptor} stanowi solidne narzędzie zarówno dla użytkowników końcowych, jak i dla twórców rozwiązań technologicznych.